\chapter{Conclusions and Future Work}
%Validating my system with an exploratory study is only the first step in a larger design philosophy. I will indicate avenues of improvement for hardware and software focusing on moving to a system on a chip (SoC) design. I will present my conclusions and suggest future steps in the research.

Throughout the process of this dissertation research new findings were always being rolled into the existing framework. The hardware platform for the WHIP sensor was developed from January to May 2014 . This design acquitted itself well through the entire process of the following trial. While some of its functions were under utilized, i.e. the Bluetooth communications, the platform demonstrates an ability to serve as a platform for future research with the ability to support more accurate ECG readings, different \spo2 measuring devices such as nasal alar sensors, and as demonstrated good low power design capabilities. 

A major limitation in the adoption of such a device is a physical constraint not a technological one. The device may be split into smaller modules in future iterations, to more closely contour the human body. This will not be a change to the design of the device, merely a dividing of the modules into different physical PCBs.

The Cellphone was effective at conveying somatic surveys to patients and allowing them to upload information. Future work may include studies on effective user interface choices to simplify the process for patients unfamiliar with cellphone technologies.